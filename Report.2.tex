\documentclass[conference]{IEEEtran}
\usepackage[backend=biber]{biblatex} 
\addbibresource{references.bib}
\title{%
 Medml lab 2 \\
  \large Measurement of Fetal head circumference using Ultrasound 
  }

\author{Pham Dinh Bao Khoi - 23BI14230}

\begin{document}

\maketitle
\section{Introduction}
The measurement of fetal head circumference (HC) is a fundamental biometric assessed throughout pregnancy to monitor fetal development and estimate gestational age \cite{intro}. In standard clinical practice, this measurement is derived from ultrasound (US) images, typically by manually tracing the skull or fitting an ellipse to the anatomy \cite{manual}. However, the process of identifying the head contour is complicated by the inherent low signal-to-noise ratio of US images, as well as borders that are often fuzzy or incomplete \cite{fuzzy}. As a result, manual contouring is an operator-dependent task prone to significant variability; specifically, intra-operator and inter-operator variability have been recorded at $\pm7$mm and $\pm12$mm respectively, which can compromise measurement accuracy \cite{sarris}.

Although various studies have proposed automating HC measurement through image segmentation algorithms \cite{li}, recent advancements in medical imaging have shifted toward the direct extraction of biomarkers—such as organ volume or area—without an intermediate segmentation step \cite{zhen}. Segmentation approaches can be computationally intensive for both training and labeling, and they remain susceptible to errors \cite{zhen}. For instance, direct volume estimation of cardiac ventricles from magnetic resonance images has been successfully demonstrated using learning-based approaches \cite{cardiac}. By utilizing Convolutional Neural Networks (CNNs), it is possible to skip the manual feature design step and learn the necessary features to perform regression directly. This regression CNN approach has already proven effective in computer vision applications, including head-pose estimation and facial landmark detection \cite{liu, sun}.




\section{EXPERIMENT}
using ntoebook Kaagle to EDA and GPU P100 to build model deep learning (updating). 
\subsection{EDA}
as you can see in the notebook
\subsection{modeling}
errrr, I have been  setting up environment for the model for a long time but not done yet.

\section{result}
It might be a late submission. BUT atleast I don't use chat gpt...




\begin{thebibliography}{99}
%%%% add
% [1] The main paper (Zhang et al.)
\bibitem{zhang2020}
J. Zhang, C. Petitjean, P. Lopez, and S. Ainouz, ``Direct estimation of fetal head circumference from ultrasound images based on regression CNN,'' \textit{Proceedings of Machine Learning Research}, vol. 121, pp. 914--922, 2020.

% [2] DatasetNinja Link
\bibitem{datasetninja}
DatasetNinja, ``Fetal Head Ultrasound Dataset.'' [Online]. Available: https://datasetninja.com/fetal-head-ultrasound\#explore

% [3] & [8] HC18 Dataset (Zenodo)
\bibitem{hc18_zenodo}
T. L. van den Heuvel, D. de Bruijn, C. L. de Korte, and B. v. Ginneken, ``Automated measurement of fetal head circumference using 2D ultrasound images,'' Zenodo, 2018. DOI: 10.5281/zenodo.1322001.

% [4] GitHub Repository
\bibitem{github_repo}
``MLMed2026 Repository.'' [Online]. Available: https://github.com/Palm-Pham/mlmed2026

% [5] ArXiv Paper (van den Heuvel et al.)
\bibitem{van2018arxiv}
T. L. A. van den Heuvel et al., ``Automated measurement of fetal head circumference using 2D ultrasound images,'' \textit{arXiv preprint arXiv:1805.00794}, 2018.

% [6] HC18 Test Set (Zenodo)
\bibitem{hc18_test}
``HC18 Test Set,'' Zenodo. [Online]. Available: https://zenodo.org/records/1327317

% [7] HC18 Challenge Leaderboard
\bibitem{hc18_challenge}
``HC18 Challenge Leaderboard.'' [Online]. Available: https://hc18.grand-challenge.org/evaluation/challenge/leaderboard/

% [9] U-Net Paper (Ronneberger et al.)
\bibitem{unet}
O. Ronneberger, P. Fischer, and T. Brox, ``U-Net: Convolutional Networks for Biomedical Image Segmentation,'' \textit{arXiv preprint arXiv:1505.04597}, 2015. DOI: 10.48550/arXiv.1505.04597

% [10] OpenCV Library
\bibitem{opencv}
G. Bradski, ``The OpenCV Library,'' \textit{Dr. Dobb’s Journal of Software Tools}, 2000.


%%%%
\bibitem{intro}
J. Zhang et al., "Direct estimation of fetal head circumference from ultrasound images based on regression CNN," \textit{Proceedings of Machine Learning Research}, vol. 121, pp. 914-922, 2020.

\bibitem{manual}
T. L. A. van den Heuvel et al., "Automated measurement of fetal head circumference using 2d ultrasound images," \textit{PLOS ONE}, vol. 13, no. 8, 2018.

\bibitem{fuzzy}
will be  Provided Source text, Section 1, Para 1.

\bibitem{sarris}
I. Sarris et al., "Intra-and interobserver variability in fetal ultrasound measurements," \textit{Ultrasound in obstetrics \& gynecology}, vol. 39, no. 3, pp. 266-273, 2012.

\bibitem{li}
J. Li et al., "Automatic fetal head circumference measurement in ultrasound using random forest and fast ellipse fitting," \textit{IEEE journal of biomedical and health informatics}, vol. 22, no. 1, pp. 215-223, 2017.

\bibitem{zhen}
X. Zhen and S. Li, "Towards direct medical image analysis without segmentation," \textit{CoRR}, abs/1510.06375, 2015.

\bibitem{cardiac}
X. Zhen et al., "Direct volume estimation without segmentation," \textit{SPIE Progress in Biomedical Optics and Imaging}, vol. 9413, 2015.

\bibitem{liu}
X. Liu et al., "3d head pose estimation with convolutional neural network trained on synthetic images," in \textit{IEEE ICIP}, pp. 1289-1293, 2016.

\bibitem{sun}
Y. Sun, X. Wang, and X. Tang, "Deep convolutional network cascade for facial point detection," in \textit{Proceedings of IEEE CVPR}, pp. 3476-3483, 2013.

\bibitem{proposal}
J. Zhang, C. Petitjean, P. Lopez, and S. Ainouz, "Direct Estimation of Fetal Head Circumference Based on Regression CNN," \textit{MIDL 2020}.

\bibitem{arch}
I will Provided Source text, Section 3.

\bibitem{hc18}
T. L. A. van den Heuvel et al., "Automated measurement of fetal head circumference using 2d ultrasound images [data set]," \textit{Zenodo}, 2018.

\bibitem{novelty}
I will Provided Source text, Section 1, Para 4.

\end{thebibliography}
    
\end{enumerate}
